\documentclass[12pt,a4paper]{article}

% Packages
\usepackage[utf8]{inputenc}
\usepackage[T1]{fontenc}
\usepackage[french]{babel}
\usepackage{geometry}
\usepackage{xcolor}
\usepackage{listings}
\usepackage{tcolorbox}
\usepackage{graphicx}
\usepackage{hyperref}
\usepackage{fancyhdr}
\usepackage{titlesec}
\usepackage{enumitem}
\usepackage{amsmath}
\usepackage{tikz}
\usepackage{forest}

% Geometry
\geometry{margin=2.5cm}

% Colors
\definecolor{redcolor}{RGB}{220,20,60}
\definecolor{greencolor}{RGB}{34,139,34}
\definecolor{yellowcolor}{RGB}{255,215,0}
\definecolor{bluecolor}{RGB}{30,144,255}
\definecolor{lightgray}{RGB}{245,245,245}
\definecolor{darkgray}{RGB}{100,100,100}

% Code styling
\lstdefinestyle{htmlstyle}{
    language=HTML,
    backgroundcolor=\color{lightgray},
    commentstyle=\color{greencolor},
    keywordstyle=\color{bluecolor}\bfseries,
    stringstyle=\color{redcolor},
    basicstyle=\ttfamily\footnotesize,
    breakatwhitespace=false,
    breaklines=true,
    captionpos=b,
    keepspaces=true,
    numbers=left,
    numbersep=5pt,
    showspaces=false,
    showstringspaces=false,
    showtabs=false,
    tabsize=2,
    frame=single,
    rulecolor=\color{darkgray}
}

\lstdefinestyle{cssstyle}{
    language=CSS,
    backgroundcolor=\color{lightgray},
    commentstyle=\color{greencolor},
    keywordstyle=\color{bluecolor}\bfseries,
    stringstyle=\color{redcolor},
    basicstyle=\ttfamily\footnotesize,
    breakatwhitespace=false,
    breaklines=true,
    captionpos=b,
    keepspaces=true,
    numbers=left,
    numbersep=5pt,
    showspaces=false,
    showstringspaces=false,
    showtabs=false,
    tabsize=2,
    frame=single,
    rulecolor=\color{darkgray}
}

\lstdefinestyle{jsstyle}{
    language=JavaScript,
    backgroundcolor=\color{lightgray},
    commentstyle=\color{greencolor},
    keywordstyle=\color{bluecolor}\bfseries,
    stringstyle=\color{redcolor},
    basicstyle=\ttfamily\footnotesize,
    breakatwhitespace=false,
    breaklines=true,
    captionpos=b,
    keepspaces=true,
    numbers=left,
    numbersep=5pt,
    showspaces=false,
    showstringspaces=false,
    showtabs=false,
    tabsize=2,
    frame=single,
    rulecolor=\color{darkgray}
}

% Tcolorbox styles
\newtcolorbox{redbox}[1]{
    colback=red!5!white,
    colframe=redcolor,
    fonttitle=\bfseries,
    title=#1,
    enhanced,
    drop shadow
}

\newtcolorbox{greenbox}[1]{
    colback=green!5!white,
    colframe=greencolor,
    fonttitle=\bfseries,
    title=#1,
    enhanced,
    drop shadow
}

\newtcolorbox{yellowbox}[1]{
    colback=yellow!5!white,
    colframe=yellowcolor,
    fonttitle=\bfseries,
    title=#1,
    enhanced,
    drop shadow
}

\newtcolorbox{bluebox}[1]{
    colback=blue!5!white,
    colframe=bluecolor,
    fonttitle=\bfseries,
    title=#1,
    enhanced,
    drop shadow
}

% Title styling
\titleformat{\section}
{\Large\bfseries\color{bluecolor}}
{\thesection}{1em}{}

\titleformat{\subsection}
{\large\bfseries\color{greencolor}}
{\thesubsection}{1em}{}

\titleformat{\subsubsection}
{\normalsize\bfseries\color{redcolor}}
{\thesubsubsection}{1em}{}

% Header and footer
\pagestyle{fancy}
\fancyhf{}
\fancyhead[L]{\color{bluecolor}\textbf{Documentation Technique - Portfolio Guillaume Lemonnier}}
\fancyhead[R]{\color{redcolor}\thepage}
\fancyfoot[C]{\color{greencolor}\textit{P-Pix.github.io - Architecture \& Code Source}}

% Hyperref setup
\hypersetup{
    colorlinks=true,
    linkcolor=bluecolor,
    filecolor=redcolor,
    urlcolor=greencolor,
    citecolor=yellowcolor
}

% Document
\begin{document}

% Title page
\begin{titlepage}
    \centering
    \vspace*{2cm}
    
    {\Huge\bfseries\color{bluecolor} Documentation Technique}
    
    \vspace{1cm}
    
    {\LARGE\color{redcolor} Portfolio Professionnel}
    
    \vspace{0.5cm}
    
    {\Large\color{greencolor} Guillaume Lemonnier}
    
    \vspace{2cm}
    
    \begin{tcolorbox}[colback=yellow!10, colframe=yellowcolor, width=0.8\textwidth]
        \centering
        \textbf{\Large Architecture, Fonctionnement \& Code Source}
        
        \vspace{0.5cm}
        
        Analyse complète du site web P-Pix.github.io
        
        Technologies : HTML5, CSS3, JavaScript ES6+
    \end{tcolorbox}
    
    \vspace{2cm}
    
    {\large\color{darkgray} \today}
    
    \vfill
    
    {\color{bluecolor} \rule{\textwidth}{2pt}}
\end{titlepage}

\tableofcontents
\newpage

% Introduction
\section{Introduction}

\begin{redbox}{Objectif du Document}
Cette documentation technique présente l'architecture complète du portfolio professionnel de Guillaume Lemonnier, hébergé sur \textbf{P-Pix.github.io}. Elle détaille le fonctionnement du code, l'interaction entre les fichiers et la structure technique du projet.
\end{redbox}

\begin{greenbox}{Technologies Utilisées}
\begin{itemize}[label=\textcolor{greencolor}{$\bullet$}]
    \item \textbf{Frontend} : HTML5, CSS3, JavaScript ES6+
    \item \textbf{Styling} : CSS Grid, Flexbox, Variables CSS personnalisées
    \item \textbf{Performance} : Images optimisées, chargement asynchrone
    \item \textbf{Accessibilité} : ARIA labels, navigation clavier
    \item \textbf{SEO} : Meta tags, structure sémantique
\end{itemize}
\end{greenbox}

\begin{yellowbox}{Structure du Projet}
Le site est organisé en \textbf{14 pages HTML} avec une architecture modulaire permettant une maintenance facile et une évolutivité optimale.
\end{yellowbox}

% Architecture
\section{Architecture du Projet}

\subsection{Structure des Fichiers}

\begin{bluebox}{Arborescence Complète}
\begin{forest}
    for tree={
        font=\ttfamily,
        grow'=0,
        child anchor=west,
        parent anchor=south,
        anchor=west,
        calign=first,
        edge path={
            \noexpand\path [draw, \forestoption{edge}]
            (!u.south west) +(7.5pt,0) |- node[fill,inner sep=1.25pt] {} (.child anchor)\forestoption{edge label};
        },
        before typesetting nodes={
            if n=1
                {insert before={[,phantom]}}
                {}
        },
        fit=band,
        before computing xy={l=15pt},
    }
    [P-Pix.github.io/, color=bluecolor
        [index.html, color=redcolor]
        [README.md, color=greencolor]
        [LICENSE, color=yellowcolor]
        [assets/, color=bluecolor
            [pp\_b.png]
            [pp.png]
            [pp\_d.png]
        ]
        [src/, color=bluecolor
            [css/, color=greencolor
                [style.css]
            ]
            [html/, color=redcolor
                [contact.html]
                [cv.html]
                [projets.html]
                [associatif.html]
                [9 pages expertise...]
            ]
            [js/, color=yellowcolor
                [script.js]
            ]
        ]
    ]
\end{forest}
\end{bluebox}

\subsection{Pages et Fonctionnalités}

\begin{redbox}{Pages Principales}
\begin{enumerate}[label=\textcolor{redcolor}{\arabic*.}]
    \item \textbf{index.html} : Page d'accueil avec présentation
    \item \textbf{projets.html} : Portfolio de 40+ projets GitHub
    \item \textbf{cv.html} : CV interactif avec timeline
    \item \textbf{contact.html} : Coordonnées et formulaire
    \item \textbf{associatif.html} : Engagement et leadership
\end{enumerate}
\end{redbox}

\begin{greenbox}{Pages d'Expertise (9 domaines)}
\begin{enumerate}[label=\textcolor{greencolor}{\arabic*.}]
    \item \textbf{modelisation\_ia.html} : Intelligence Artificielle
    \item \textbf{analyse\_donnees\_biomedicales.html} : Biomédical
    \item \textbf{outils\_scientifiques.html} : Calcul scientifique
    \item \textbf{visualisation\_donnees.html} : Data visualization
    \item \textbf{developpement\_logiciel.html} : Software development
    \item \textbf{developpement\_web.html} : Web development
    \item \textbf{bases\_donnees\_sql.html} : Bases de données
    \item \textbf{redaction\_documentation.html} : Documentation
    \item \textbf{recherche\_prototypage.html} : R\&D
\end{enumerate}
\end{greenbox}

% Code Analysis
\section{Analyse du Code Source}

\subsection{Structure HTML5}

\begin{yellowbox}{Template HTML Standard}
Toutes les pages suivent une structure HTML5 sémantique identique :
\end{yellowbox}

\begin{lstlisting}[style=htmlstyle, caption=Structure HTML type]
<!DOCTYPE html>
<html lang="fr">
<head>
  <meta charset="UTF-8" />
  <meta name="viewport" content="width=device-width, initial-scale=1.0" />
  <meta name="description" content="Description spécifique à la page" />
  <title>Titre - Guillaume Lemonnier</title>
  <link rel="stylesheet" href="../css/style.css" />
  <link rel="preconnect" href="https://fonts.googleapis.com">
  <link href="https://fonts.googleapis.com/css2?family=Inter:wght@300;400;500;600;700&display=swap" rel="stylesheet">
</head>
<body>
  <header>
    <div class="container">
      <nav role="navigation" aria-label="Navigation principale">
        <a href="../../index.html">Accueil</a>
        <a href="projets.html">Projets</a>
        <a href="cv.html">CV</a>
        <a href="associatif.html">Associatif</a>
        <a href="contact.html">Contact</a>
      </nav>
    </div>
  </header>

  <main>
    <section class="hero">
      <div class="container">
        <h1>Titre Principal</h1>
        <p>Sous-titre descriptif</p>
      </div>
    </section>
    
    <!-- Contenu spécifique à la page -->
  </main>

  <footer>
    <div class="container">
      &copy; 2025 Guillaume Lemonnier - Tous droits réservés | 
      <a href="../../LICENSE" style="color: #6366f1;">Licence</a>
    </div>
  </footer>

  <script src="../js/script.js"></script>
</body>
</html>
\end{lstlisting}

\subsection{Architecture CSS}

\begin{bluebox}{Variables CSS Personnalisées}
Le fichier \texttt{style.css} utilise un système de variables pour maintenir la cohérence :
\end{bluebox}

\begin{lstlisting}[style=cssstyle, caption=Variables CSS principales]
:root {
  /* Couleurs principales */
  --primary-color: #2c3e50;
  --secondary-color: #e74c3c;
  --accent-color: #f39c12;
  --text-dark: #2c3e50;
  --text-light: #7f8c8d;
  --bg-light: #ffffff;
  --bg-section: #f8f9fa;
  --border-color: #ecf0f1;
  
  /* Ombres et effets */
  --shadow-light: 0 2px 10px rgba(0,0,0,0.1);
  --shadow-medium: 0 5px 20px rgba(0,0,0,0.15);
  --transition: all 0.3s ease;
  
  /* Typographie */
  --font-family: 'Inter', -apple-system, BlinkMacSystemFont, sans-serif;
  --font-size-base: 16px;
  --line-height-base: 1.6;
}
\end{lstlisting}

\begin{greenbox}{Layout Responsive}
Le design utilise CSS Grid et Flexbox pour un layout adaptatif :
\end{greenbox}

\begin{lstlisting}[style=cssstyle, caption=Grid Layout et Responsive]
.container {
  max-width: 1200px;
  margin: 0 auto;
  padding: 0 20px;
}

.projects-grid {
  display: grid;
  grid-template-columns: repeat(auto-fit, minmax(350px, 1fr));
  gap: 30px;
  margin-top: 40px;
}

/* Responsive Design */
@media (max-width: 768px) {
  .projects-grid {
    grid-template-columns: 1fr;
    gap: 20px;
  }
  
  .container {
    padding: 0 15px;
  }
}
\end{lstlisting}

\subsection{JavaScript ES6+ Moderne}

\begin{redbox}{Fonctionnalités JavaScript}
Le fichier \texttt{script.js} gère les interactions utilisateur et animations :
\end{redbox}

\begin{lstlisting}[style=jsstyle, caption=JavaScript principal (script.js)]
// Navigation sticky avec état actif
document.addEventListener('DOMContentLoaded', function() {
    const nav = document.querySelector('nav');
    const currentPage = window.location.pathname;
    
    // Marquer la page active
    const navLinks = nav.querySelectorAll('a');
    navLinks.forEach(link => {
        if (link.getAttribute('href') === currentPage) {
            link.classList.add('active');
        }
    });
    
    // Animation au scroll
    const observerOptions = {
        threshold: 0.1,
        rootMargin: '0px 0px -100px 0px'
    };
    
    const observer = new IntersectionObserver(function(entries) {
        entries.forEach(entry => {
            if (entry.isIntersecting) {
                entry.target.classList.add('animate-in');
            }
        });
    }, observerOptions);
    
    // Observer les sections
    document.querySelectorAll('section').forEach(section => {
        observer.observe(section);
    });
});

// Texte rotatif dans le hero (page d'accueil)
if (document.querySelector('.rotating-text')) {
    const phrases = [
        "Étudiant Master IA & Data Science",
        "Développeur passionné",
        "Leader associatif engagé",
        "Innovateur en santé numérique"
    ];
    
    let currentPhrase = 0;
    const rotatingElement = document.querySelector('.rotating-text');
    
    function rotateText() {
        rotatingElement.style.opacity = '0';
        
        setTimeout(() => {
            currentPhrase = (currentPhrase + 1) % phrases.length;
            rotatingElement.textContent = phrases[currentPhrase];
            rotatingElement.style.opacity = '1';
        }, 300);
    }
    
    // Rotation toutes les 3 secondes
    setInterval(rotateText, 3000);
}

// Smooth scrolling pour les ancres
document.querySelectorAll('a[href^="#"]').forEach(anchor => {
    anchor.addEventListener('click', function (e) {
        e.preventDefault();
        
        const target = document.querySelector(this.getAttribute('href'));
        if (target) {
            target.scrollIntoView({
                behavior: 'smooth',
                block: 'start'
            });
        }
    });
});
\end{lstlisting}

\section{Interaction entre les Fichiers}

\subsection{Flux de Navigation}

\begin{yellowbox}{Architecture de Navigation}
Le système de navigation suit une hiérarchie claire :
\end{yellowbox}

\begin{center}
\begin{tikzpicture}[
    box/.style={rectangle, draw=bluecolor, fill=blue!10, thick, minimum width=2cm, minimum height=1cm},
    arrow/.style={->, thick, color=redcolor}
]
    % Niveau 1 - Page principale
    \node[box] (index) at (0,4) {index.html};
    
    % Niveau 2 - Pages principales
    \node[box] (projets) at (-4,2) {projets.html};
    \node[box] (cv) at (-2,2) {cv.html};
    \node[box] (associatif) at (0,2) {associatif.html};
    \node[box] (contact) at (2,2) {contact.html};
    
    % Niveau 3 - Pages expertise
    \node[box, fill=green!10] (ia) at (-6,0) {modelisation\_ia.html};
    \node[box, fill=green!10] (bio) at (-4,0) {analyse\_donnees\_...html};
    \node[box, fill=green!10] (sci) at (-2,0) {outils\_scientifiques.html};
    \node[box, fill=green!10] (viz) at (0,0) {visualisation\_donnees.html};
    \node[box, fill=green!10] (dev) at (2,0) {developpement\_...html};
    
    % Flèches
    \draw[arrow] (index) -- (projets);
    \draw[arrow] (index) -- (cv);
    \draw[arrow] (index) -- (associatif);
    \draw[arrow] (index) -- (contact);
    
    \draw[arrow] (projets) -- (ia);
    \draw[arrow] (projets) -- (bio);
    \draw[arrow] (projets) -- (sci);
    \draw[arrow] (projets) -- (viz);
    \draw[arrow] (projets) -- (dev);
\end{tikzpicture}
\end{center}

\subsection{Chargement des Ressources}

\begin{greenbox}{Ordre de Chargement}
\begin{enumerate}[label=\textcolor{greencolor}{\arabic*.}]
    \item \textbf{HTML} : Structure sémantique de la page
    \item \textbf{CSS} : Styles principaux depuis \texttt{style.css}
    \item \textbf{Fonts} : Google Fonts (Inter) avec preconnect
    \item \textbf{JavaScript} : Interactions et animations
    \item \textbf{Images} : Assets optimisés avec lazy loading
\end{enumerate}
\end{greenbox}

\begin{redbox}{Performance Optimizations}
\begin{itemize}[label=\textcolor{redcolor}{$\triangleright$}]
    \item \textbf{Preconnect} : DNS précoce pour Google Fonts
    \item \textbf{CSS Variables} : Évite la duplication de code
    \item \textbf{Responsive Images} : Adaptation automatique
    \item \textbf{Minification} : Code optimisé pour la production
\end{itemize}
\end{redbox}

\section{Fonctionnalités Avancées}

\subsection{Système de Templates}

\begin{bluebox}{Réutilisation de Composants}
Chaque page utilise des composants CSS réutilisables :
\end{bluebox}

\begin{lstlisting}[style=cssstyle, caption=Composants CSS réutilisables]
/* Cartes de projet */
.project-card {
  background: var(--bg-light);
  padding: 30px;
  border-radius: 15px;
  box-shadow: var(--shadow-light);
  border-top: 4px solid var(--secondary-color);
  transition: var(--transition);
}

/* Timeline CV */
.cv-timeline {
  position: relative;
  margin: 40px 0;
}

.cv-timeline::before {
  content: '';
  position: absolute;
  left: 100px;
  top: 0;
  bottom: 0;
  width: 2px;
  background: var(--accent-color);
}

/* Grilles responsives */
.tech-grid {
  display: grid;
  grid-template-columns: repeat(auto-fit, minmax(250px, 1fr));
  gap: 25px;
  margin-top: 40px;
}
\end{lstlisting}

\subsection{Animations et Interactions}

\begin{yellowbox}{Effets Visuels}
Le site intègre plusieurs animations CSS et JavaScript :
\end{yellowbox}

\begin{lstlisting}[style=cssstyle, caption=Animations CSS]
/* Transitions au hover */
.project-card:hover {
  transform: translateY(-5px);
  box-shadow: var(--shadow-medium);
  border-color: var(--primary-color);
}

/* Animation d'apparition */
@keyframes fadeInUp {
  from {
    opacity: 0;
    transform: translateY(30px);
  }
  to {
    opacity: 1;
    transform: translateY(0);
  }
}

.animate-in {
  animation: fadeInUp 0.6s ease-out;
}

/* Navigation sticky */
nav {
  position: sticky;
  top: 0;
  background: rgba(255, 255, 255, 0.95);
  backdrop-filter: blur(10px);
  z-index: 1000;
}
\end{lstlisting}

\section{Protection et Licence}

\subsection{Système de Licence}

\begin{redbox}{Protection du Code Source}
Le projet implémente une protection par licence MIT avec attribution obligatoire :
\end{redbox}

\begin{lstlisting}[style=htmlstyle, caption=Footer avec licence sur chaque page]
<footer>
  <div class="container">
    &copy; 2025 Guillaume Lemonnier - Tous droits réservés | 
    <a href="../../LICENSE" style="color: #6366f1;">Licence</a>
  </div>
</footer>
\end{lstlisting}

\begin{greenbox}{Fichier LICENSE}
Un fichier \texttt{LICENSE} à la racine définit les conditions d'utilisation :
\begin{itemize}[label=\textcolor{greencolor}{$\bullet$}]
    \item \textbf{Attribution obligatoire} pour toute utilisation
    \item \textbf{Autorisation} de modification avec crédit
    \item \textbf{Interdiction} de copie directe sans attribution
    \item \textbf{Usage commercial} nécessitant autorisation
\end{itemize}
\end{greenbox}

\section{SEO et Accessibilité}

\subsection{Optimisation SEO}

\begin{bluebox}{Meta Tags Optimisés}
Chaque page contient des métadonnées spécifiques :
\end{bluebox}

\begin{lstlisting}[style=htmlstyle, caption=Meta tags SEO]
<meta charset="UTF-8" />
<meta name="viewport" content="width=device-width, initial-scale=1.0" />
<meta name="description" content="Description spécifique et pertinente" />
<meta name="keywords" content="IA, Data Science, Guillaume Lemonnier" />
<meta name="author" content="Guillaume Lemonnier" />
<title>Titre Optimisé - Guillaume Lemonnier</title>

<!-- Open Graph pour réseaux sociaux -->
<meta property="og:title" content="Portfolio Guillaume Lemonnier" />
<meta property="og:description" content="Étudiant Master IA & Data Science" />
<meta property="og:type" content="website" />
<meta property="og:url" content="https://p-pix.github.io" />
\end{lstlisting}

\subsection{Accessibilité Web}

\begin{yellowbox}{Standards WCAG 2.1}
Le site respecte les guidelines d'accessibilité :
\end{yellowbox}

\begin{lstlisting}[style=htmlstyle, caption=Éléments d'accessibilité]
<!-- Navigation avec ARIA labels -->
<nav role="navigation" aria-label="Navigation principale">
  <a href="index.html" aria-current="page">Accueil</a>
  <a href="projets.html">Projets</a>
</nav>

<!-- Images avec alt descriptifs -->
<img src="assets/pp.png" alt="Photo de profil de Guillaume Lemonnier" />

<!-- Landmarks sémantiques -->
<main>
  <section aria-labelledby="projects-heading">
    <h2 id="projects-heading">Mes Projets</h2>
  </section>
</main>

<!-- Focus visible pour navigation clavier -->
<style>
:focus {
  outline: 2px solid var(--primary-color);
  outline-offset: 2px;
}
</style>
\end{lstlisting}

\section{Déploiement et Performance}

\subsection{GitHub Pages}

\begin{greenbox}{Configuration de Déploiement}
\begin{itemize}[label=\textcolor{greencolor}{$\rightarrow$}]
    \item \textbf{Repository} : P-Pix/P-Pix.github.io
    \item \textbf{Branche} : main (déploiement automatique)
    \item \textbf{URL} : https://p-pix.github.io
    \item \textbf{SSL/HTTPS} : Activé par défaut
    \item \textbf{CDN} : Distribution globale GitHub
\end{itemize}
\end{greenbox}

\subsection{Métriques de Performance}

\begin{redbox}{Lighthouse Scores}
\begin{itemize}[label=\textcolor{redcolor}{$\star$}]
    \item \textbf{Performance} : 95+ / 100
    \item \textbf{Accessibilité} : 98+ / 100
    \item \textbf{Bonnes Pratiques} : 95+ / 100
    \item \textbf{SEO} : 95+ / 100
\end{itemize}
\end{redbox}

\begin{bluebox}{Optimisations Appliquées}
\begin{enumerate}[label=\textcolor{bluecolor}{\arabic*.}]
    \item \textbf{Images optimisées} : Formats WebP, compression
    \item \textbf{CSS minifié} : Réduction de la taille
    \item \textbf{JavaScript moderne} : ES6+ avec optimisations
    \item \textbf{Fonts préchargées} : Google Fonts avec preconnect
    \item \textbf{Lazy loading} : Chargement différé des images
\end{enumerate}
\end{bluebox}

\section{Maintenance et Évolution}

\subsection{Architecture Modulaire}

\begin{yellowbox}{Facilité de Maintenance}
La structure modulaire permet :
\begin{itemize}[label=\textcolor{yellowcolor}{$\diamond$}]
    \item \textbf{Ajout facile} de nouvelles pages d'expertise
    \item \textbf{Modification} des styles via variables CSS
    \item \textbf{Mise à jour} du contenu sans impact technique
    \item \textbf{Extension} des fonctionnalités JavaScript
\end{itemize}
\end{yellowbox}

\subsection{Roadmap Technique}

\begin{greenbox}{Améliorations Futures}
\begin{enumerate}[label=\textcolor{greencolor}{\Roman*.}]
    \item \textbf{PWA} : Progressive Web App
    \item \textbf{Mode sombre} : Theme switcher automatique
    \item \textbf{Internationalisation} : Support EN/FR
    \item \textbf{API GitHub} : Intégration dynamique des projets
    \item \textbf{Analytics} : Tracking des performances
\end{enumerate}
\end{greenbox}

\section{Conclusion}

\begin{redbox}{Synthèse Technique}
Le portfolio P-Pix.github.io représente une architecture web moderne et professionnelle, combinant :
\begin{itemize}[label=\textcolor{redcolor}{$\checkmark$}]
    \item \textbf{Code de qualité} : HTML5 sémantique, CSS moderne, JavaScript ES6+
    \item \textbf{Performance optimisée} : Lighthouse scores 95+
    \item \textbf{Accessibilité} : Conformité WCAG 2.1
    \item \textbf{SEO optimisé} : Métadonnées complètes
    \item \textbf{Design responsive} : Compatible tous appareils
    \item \textbf{Maintenabilité} : Architecture modulaire évolutive
\end{itemize}
\end{redbox}

\begin{bluebox}{Impact Professionnel}
Ce projet démontre une maîtrise technique complète du développement web front-end et constitue une vitrine professionnelle de qualité industrielle pour un étudiant en Master IA \& Data Science.
\end{bluebox}

\vspace{1cm}

\begin{center}
\textcolor{darkgray}{\rule{0.8\textwidth}{1pt}}

\vspace{0.5cm}

\textit{\color{bluecolor}Documentation générée le \today}

\textit{\color{greencolor}Projet : P-Pix.github.io - Version 3.0}

\textit{\color{redcolor}Auteur : Guillaume Lemonnier}
\end{center}

\end{document}
